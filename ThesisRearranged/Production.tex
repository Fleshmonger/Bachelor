\chapter{Production}
\label{ch:production}
This chapter concerns the creation of units including the architecture for its management. First we explain the StarCraft terms of units and their production.

A \emph{unit} is any player controlled entity, structure or not. Resources are also units, controlled by a neutral "player", but are both immobile and indestructible. Some spell effects, like the Terran nuke, are oddly also classified as units but they are neither selectable nor controllable. The term unit rarely includes these however.

\emph{Production} here refers to the player controlled act of creating a new unit. Within StarCraft, producing a structure is called \emph{building} and \emph{training} if it is a non-structure. Once a unit is created, there is a duration where it is \emph{constructing} before becoming \emph{complete}. A unit cannot execute player commands and has no abilities during construction. Constructing structure are placed in the world when they are built, while non-structure are hidden within their constructor. When a non-structure is complete, it appears at the nearest free space around its constructor. A unit consumes resources upon production, including supply requirements. A constructing non-structure can be canceled for a full refund.

BWAPI notifies the agent whenever a unit is created or completed in separate events, which can be used to update internal data-structures.

As an additional note, BWAPI divides the map into \emph{regions}, which contains resource and base locations. They are important because the regions border each other at \emph{chokepoints}, which are either ramps or bottlenecked passages. Usually a region contains none or a single base location and resource cluster, however some regions contain multiple clusters with overlapping base locations. Because of these qualities, regions are very useful in both combat and economy, and is used throughout the implementation. They are needed for expansions, explained in Section \ref{sec:buildExpansions}. BWTA redefines the regions to contain the optimal base locations in expansion.

In the following sections we first briefly describe our location in the hierarchy. Then we begin with training of non-structure, moving on to building structures. Finally we cover expansions and expanding.

\section{Production Architecture}
The \emph{Accountant} module is used for keeping internal track of spent resources and scheduled units. Commanding a unit to build or train will not spend resources until at least the next frame. Therefore it is required to keep internal records of these, otherwise different modules might spend the same resources twice. Unit scheduling is useful in later frames, such that modules do not request the same unit multiple times.

Training is handled by the \emph{Recruiter} module, while building is done by the \emph{Architect}. These are separate modules as the two jobs are very different in implementation. While the Recruiter is quite low in the hierarchy, but the Architect is not since it requires workers from the Landlord module. Both are superior to the Accountant.

There is a third module, the \emph{Morpher}, which handles the rare case of a unit that executes a \emph{morph} command. A morphing unit transforms from one type to another, which is most often seen in case of building refineries. Counter-intuitively, the geyser morphs into a refinery and changes ownership to the constructing player, but the refinery is still built by a worker with a build command. The Morpher exclusively monitors morphing units, while the Architect handles the building.

\section{Training Units}
\label{sec:training}
The \emph{Recruiter} needs to be low in the hierarchy, as it needs to be accessed by many other modules. It does not contain an independent AI, acting only through certain events and method calls. The current implementation is generic such that it can handle any train-able unit in the game.

To train units, the steps are twofold: issue the train command to a relevant, available trainer and then monitor the construction.

	\subsection*{Commanding Trainer}
	Some units can train others, such as the depots that train workers. It is almost exclusively structures that can train. A unit can only train one at a time, although multiple can be queued up. Every unit is trained by at most one other unit, so a graph of trainers and trainees would be a \emph{forest} structure.
	
	Notice that it is inefficient to queue training, as this will lock resources. Instead, issuing the train command whenever the trainer is available will keep resources free without resorting to cancellation. It would be better to implement an internal queue if one was needed.
	
	To command the training we require the trainer. For this we would like to keep records of all units capable of training. BWAPI inherently contains the trainer of any given unit type, so using a dictionary of trainers with their types as keys is sufficient and allows fast queries. From this we can search through the set of trainers until we find one that is available and then send the train command. The evaluation of trainers involve verifying the current existence and control of the unit, and ensuring it is not currently training another unit or has been commanded to in this frame.
	
	Searching through the dictionary is logarithmic and iteration through the matching trainers is linear. Evaluating a trainer is constant. Given $n$ trainers and $m$ trainer matches, the time complexity for training is $\log(n) + m$, which could be improved to $m$ with hashing. Neither $n$ or $m$ are usually very large, but $m$ especially is only in the range of zero to four. Insertions and deletions of trainers are both logarithmic to their size.

	\subsection*{Monitoring Trainee}
	Other modules require to know which units are currently scheduled when planning their actions. Units could also be destroyed before completion, in which case other modules might need to reschedule.
	
	Once the train command has been issued, the Accountant is notified of the costs and the type.
	
	Upon receiving the event that a unit has been created, the Recruiter frees the resource costs from the Accountant, as the game state by now has withdrawn the costs itself. The Recruiter inserts the unit into a set of incomplete units. When a unit is completed, the Recruiter is notified and removes the unit from the set and from the Accountant. The same happens if the unit is destroyed, which occurs if the trainer was destroyed.
	
	Inserting and removing from the set is logarithmic to the size. The set is never expected to be very large.

\section{Building Structures}
Like training units, building structures is done in two steps: the structure is built and then it is constructed. Contrary to training, structures require both workers and a location to be built. Workers are the only units capable of building structures, and must move to the build location to do so. Terran workers must stay and construct structures until they are complete and Zerg workers are destroyed upon building, while Protoss require neither.

However, all Protoss structures must be in close proximity to a Pylon when built, and stops functioning without. The only structure exempt from is the depot and the pylon itself. This only becomes important in later stages of the game where Players might have to carefully manage their space. The agent is not expected to build enough structures for this to be an issue, so it is mostly disregarded. A Protoss structure is considered \emph{powered} when in vicinity of a pylon.

	\subsection*{Structure Placement}
	The Protoss player must place at least one pylon in any area he wishes to build in, and could distance pylons in a base to maximize coverage. Alternatively, if only a few pylons power a structure it creates a liability. The pylon could easily be destroyed to disable the structure, which is useful if it is a defensive structure or unit trainer. Fortunately, we expect the agent's bases to be compact enough such that we can place pylons somewhat arbitrarily and still manage to fit needed structures in the region. It is important however to place at least one pylon in a base before other structures can be built, but this is solved by the strategy modules.
	
	Unless a building location is specified, the easiest solution is placing a structure as close as possible to the base location in the desired region. All the workers in the region are usually at the mineral fields by the depot, and therefore will not be far from the placement. The workers and depot become sheltered by the structures, such that the opponents troops are forced to move through bottlenecks to get the workers. Even if a region only has one exit, it might be desirable to spread structures as flying units can attack from any angle. Finally, the solution is easily implemented.
	
	There are some exceptions. Depots will always be placed in new base locations and refineries can only be built on of vespene geysers. In both of these cases the building location will be specified by the superior module ordering the structure.

	The Architect attempts all locations in the map in a spiral pattern around the depot, and returns the first location that is available. This is simple, usually cheap but very expensive asymptotically, as it will be linear to the map size. However we never expect to visit anywhere near all the locations, since few tiles in the map cannot be built upon. To determine availability of a location, it must be clear of units and the tile terrain itself must be able to be built upon. Both of these are handled by BWAPI. Note that this solution places structures in a square pattern, which is not actually the closest to the base location except in \emph{taxicap geometry}.

	Placing structures too close can block passage between them, especially for larger units. The hitbox is arbitrarily different between otherwise equal sized structures in StarCraft, such that some combinations next to each other will block some units but not necessarily all. This is because the structure hitbox usually does not cover its location completely, allowing some leeway for smaller units.
	
	By placing structures at least one tile from each other, we ensure all units can easily pass through. This is done by keeping a map of all owned structure locations, where their dimensions are increased by one tile in all directions. If a build location overlaps any of the occupied tiles it is determined as not available. Registering new structures and querying this is constant time operations, but the space used is linear to the map size. It could be improved by only keeping a map for regions we have bases in, although not an asymptotic improvement.
	
	Additionally the Architect avoids placing structures between resources and depots to avoid obstructing workers. By drawing the smallest rectangle including the depot location, geyser(s) and mineral fields, the Architect avoids blocking workers by not placing structures within this. Building the rectangle is linear to the amount of items in it and querying it is constant, but we expect the amount of items to be low (ten or less). This structure map is handled by the \emph{Base Manager} module, which the Architect is superior of.
	
	To summarize, if a location is not specified, structures are placed in a spiral pattern around the depot, distanced by one tile from others and outside the gathering-zone.

	\subsection*{Aquiring Builder}
	Recall the Task Master module from Chapter \ref{ch:resources}. Since all workers are separated in base locations, we can easily pick a worker from the region the building location is. As the task master marks the jobs of all workers, we can pick a worker from either idle, mineral mining or gas harvesting (in that order). This way we do not interrupt other possibly important tasks. The Architect searches through the groups until it finds a worker that is not carrying resources, which then becomes the builder and is tasked as such in the Task Master.
	
	The time complexity is linear to the amount of workers in these groups. However, it is probable that less than half the workers are returning resources at any given moment, so it is expected that we only need to check two workers before finding a viable candidate. In case there are idle workers, the operation is constant.
	
	This solution does not work if we have no workers in the region. This is the case when expanding, where the worker must be specified by a superior module. This is the case when the \emph{Settler} module builds an expansion, explained in Section \ref{sec:expansions}.
	
	\subsection*{Executing Command}
	Once the builder has been retrieved, it will be commanded to build at the specified location. A player cannot build in hidden terrain, so the worker will first be moved closer to the target location in this case. When the entire placement has been revealed the builder is commanded to build the structure.
	
	The build order will then be stored in a set, containing the structure type, builder and location. Every frame, the Architect verifies the validity of the build orders, including verifying the builder has not been re-tasked. It also reissues commands to builders if required. As with monitoring trainees, incomplete structures are important to keep for the same reasons. In case of structures, there are more things that could go wrong which would incur cancellation of a scheduled structure. The builder might be destroyed or the build location might prove to be invalid upon being revealed.
	
	All invalid orders are removed, forcing superior modules to reissue the build order. This is desired compared to repairing the order, as the build order might no longer be required.
	
	\subsection*{Monitoring Constructions}
	Since Protoss structures auto-construct, the implementation is identical to monitoring trainees with the same time and space complexities.
	
	As mentioned prior however, refineries are handled by the Morpher module. BWAPI is notified when a unit morphs, upon which it is inserted in a set of incomplete morphs. A morphing unit is not considered constructing, although it is incomplete. When a unit is finished morphing, no event is called, therefore forcing the Morpher to verify all morphs every frame. However, there are very few expected morphing units at any given moment.

\section{Building Expansions}
\label{sec:buildExpansions}
Recall that \emph{expansions} are additional depots beyond the initial one, built at new resource clusters. As explained in the chapter \ref{ch:resources}, it is often profitable and sometimes necessary to expand resource harvesting to new locations. To avoid transporting cargo all the way between regions, players have to build depots near resources they wish to harvest.

When BWTA analyzes a map, it marks all viable depot locations. These are positions in which a depot will be at optimal distance from nearby minerals and geysers. Usually every resource cluster only has one of these positions, but some locations may contain overlapping depots. Regions rarely have more than one base location. Given a region, we can obtain all internal base locations, including these depot locations.

This significantly simplifies the construction of expansions, however the base location to settle must still be determined.

The \emph{Settler} module is responsible for all expansion behavior, and is also used to determine if expanding is possible. To build an expansion we need to specify a location and builder. These are handled by the Settler module, while the rest is done by the usual build procedure in the Architect.

	\subsection*{Expansion Placement}
	To determine whether a base location is fit to expand it must not already be expanded upon or contain enemy forces. Additionally, it must be reachable by non-flying units and have a path of regions without enemies. If these are satisfied, we consider the base location available for expansion.
	
	From the starting region, we visit neighbor regions recursively in order of proximity by using a priority queue, where the lowest priority is picked first. Initially the start region with priority 0 is inserted. When we visit a region, we visit all contained base locations. If any are available, we return one as the target location of expansion. In case none are available, but the region is unoccupied by enemy forces, we add all unvisited neighbors to the queue. The priority is equal to the distance between the two regions centers plus the current regions priority. If the region does have enemy units within, its neighbors are not added to the queue. The algorithm terminates when the queue is empty or an available candidate has been found.

	This solution satisfies our expansion requirements, and additionally includes the starting region should the depot be lost. It is very crudely implemented however, and quite expensive. Determining whether a region is occupied involves checking all enemy units. Given $r$ regions and $n$ enemy units, the time complexity is $rn$. This could be optimized by first marking all occupied regions before the recursion.
	
	The enemy units are retrieved from the \emph{Archivist} module, detailed in Chapter \ref{ch:exploration}.
	
	There could exists more factors when determining expansion locations beyond distance. Resource quantity or defensibility are also important, but are not considered here. One could favor regions distant to the opponent for added defense or alternatively closer for better map control.

	\subsection*{Expansion Builder}
	An expansion is often the first structure in a region, and therefore the player must acquire a builder elsewhere. The Architect only attempts to pick workers from the region the location is within, so the Settler must specify a worker beforehand.
	
	In this case, we just pick a free worker from the main base. There could be workers closer to the destination in other regions, and in this case the solution is sub-optimal. Recursively checking neighbors in order of proximity would result in an approximation of the closest worker. It was not a priority however as the improvement would be insignificant.
	
	The solution is cheap, and costs the usual to pick a worker. A worker considered available is either idle or tasked with gathering.