\chapter{Introduction}
Developing artificial intelligence in \emph{real-time strategy} (RTS) games offers the challenge of limitless decision space, incomplete world information and varied strategies in a shifting meta-game. A computer-controlled player, also called a \emph{bot}, must both be efficient and effective.

StarCraft: Brood War, released by Blizzard Entertainment in 1998, is one such RTS game that has been of much focus in AI research. In addition to usual RTS elements, StarCraft is an asymmetric game. There are three different factions to play as and many asymmetric battlefields. Since the advent of the unofficial \emph{Brood War Application Interface} (BWAPI), people have been able to develop their own bots for the game. This has spawned a few tournaments where the goal is developing the best StarCraft player AI. Even though both the game and API have become dated, the tournament scene is still very active.

The challenges posed to an AI in the game are very difficult to solve both efficiently and optimally, such that even the best current bots are mediocre compared to competitive human players.

This thesis focuses on developing a competitive bot and investigates how contemporary bots solve the inherent challenges.

Chapter \ref{ch:process} explains the project process. Chapter \ref{ch:starcraft} briefly covers the basics of RTS games and StarCraft for those unfamiliar with the franchise. Chapter \ref{ch:related} discusses the challenges in more detail while providing some related works, where Chapter \ref{ch:design} discusses this project's gameplay and design strategy. Chapters \ref{ch:resources}, \ref{ch:production}, \ref{ch:exploration}, \ref{ch:combat} and \ref{ch:strategy} covers the different parts of the agent and its solutions. Chapter \ref{ch:results} presents and discusses the bots results against others.