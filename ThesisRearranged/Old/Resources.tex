%Contains base and resource management, specifically the TaskMaster.
\chapter{Resources and Bases}
In early- and mid-game, \emph{economic advantage} is the most important aspect of StarCraft. This is when you have a higher resource input than your opponent, and happens when you focus more on resources than them, or they sacrifice long term advantage over short term. The point of this is if you eclipse your opponent in resources, you can perform worse in combat and still end up on top. You can send more units, recover them faster and expand economy quicker, retaining your advantage.

There are three resources in StarCraft. Minerals are mined from mineral fields which usually are in clusters of 8-12 by each base location. Everything you produce costs minerals, so it is the most important resource. They are harvested by your workers, which bring it back to a nearby resource depot in batches of six. Usually there are 2-3 workers on each field depending on the distance from the depot. \emph{Vespene Gas} or simply gas is harvested from \emph{refineries}. These must be built upon \emph{Vespene Geysers} of which there are one or two by placed by each base location. It is harvested in batches of eight with a max of three workers per refinery at optimal depot distance. Advanced structures, units and all technologies depend on gas, and a decision with major impact is when a player starts harvesting gas. The immediate cost of the refinery and lost mineral gathering is a liability against fast rushing strategies. The last resource, supply, is reserved by units. Every unit reserves some amount of supply which is released upon their destruction. The only way to secure more is building \emph{supply depots}, \emph{pylons} or \emph{overlords} respectively for Terran, Protoss and Zerg. Each of these add some amount of supply, which is removed if they are destroyed. A player can happen to use more supply than they have, but will be unable to build more units until more supply is acquired.

Only one worker can be active at a mineral field or refinery.

StarCraft has a built in worker gathering AI. Workers will automatically return cargo from resources (unless they are interrupted). When gathering minerals, the workers will move to another mineral field if the current one is occupied. When gathering gas, they will wait until the refinery is unoccupied and then gather.

TODO TaskMaster module

\section{Mining Minerals}
The simplest mineral gathering implementation is ordering idle gatherers to mine some arbitrary mineral. At some point, the built-in AI will ensure the workers are optimally scattered. It will however not be scattered immediately, and some workers will be very inefficient while moving from mineral to mineral. A simple but effective addition would be to scatter the initial workers.

By maintaining a queue of minerals, we can optimally scatter the workers. The first element of the queue is the mineral with fewest workers and the one in the back has the most. By continually assigning new workers to the first element and moving it to the back, we maintain a queue where the last mineral has at most one more worker than the first. Removing workers however requires finding the mineral in the queue, and should be done sparingly. Therefore, it is assumed any building or defending activity will be short and temporary, and workers assigned such will not be removed from the scattering. This might result in an ineffective scattering at some points. It is not clear however if optimizing the scattering at all times results in optimal resource output, as workers might be moved between minerals too often, resulting in less time mining.

If we maintain a dictionary of workers and their targets, we can assign new ones in constant time and retrieving targets in logarithmic time. This could be improved to amortized constant time with hashing. Removing a worker however requires a search through the queue which is linear time.

TODO alternatives.

\section{Harvesting Gas}	
Gas harvest is very much like mineral mining, however it is simpler. It is agreed upon that three workers are the optimal amount per refinery. The implementation reflects this, and is otherwise identical to mineral mining.

Bases have between zero to two geysers, but in no AI tournament maps are there more than one per base location.

\section{Building Supply}
A player can have a maximum of 200 supply. The simplest supply implementation is to build supply units whenever the supply limit has been reached while it is below the maximum threshold. This will however throttle production while the supply structure is being constructed, which is inefficient. The optimal solution would be predicting supply needed in the near future and interlace supply production with unit production such that all orders are completed the earliest. Observe however that this is alike the \emph{Job shop scheduling} problem but more complex with the additional factor of resources. As such the problem is NP-complete, and even then resource gathering rates must be predicted for optimal solutions.

A simple solution which is implemented in the current bot is to build supply units when released supply is below a set threshold. This threshold is somewhat arbitrarily set according to testing. A more dynamic solution would be to set the threshold to the amount of supply used if all production facilities built a unit, assuming the unit type can be predicted. This solution will build supply units that are completed before they are needed, but might not result the optimal amount of units completed at any given time (impacting economy with relation to workers). The worst problem, production throttling, is avoided however.

\section{Managing gathering}
Usually a Protoss player should not stop building workers until late game, but some openings require a temporary pause. Therefore the solution is just greedily building workers while no build order is being executed.

Two workers per mineral is the optimal amount. The third worker will have diminishing returns and the fourth would have no impact. A third worker is optimal if the mineral is far away from the depot. Three workers on every refinery is the standard, regulated only according to need. Usually a refinery will be at max capacity, unless the player has lost too many workers.

TODO Multiple bases.

The Production chapter will cover how and where we build these expansions.

TBD Mineral to gas ratio.

TODO Maynard Slide