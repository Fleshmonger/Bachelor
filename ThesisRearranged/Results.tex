\chapter{Results}
\label{ch:results}
The bot was submitted to the Student StarCraft AI Tournament (SSCAIT). While the tournament itself has not started yet, the bots still play against each other in live-streamed matches. There is a scoreboard where bots are placed accodingly to their win-rate, with total wins and losses marked. As mentioned in Section \ref{sec:expanding} there are three versions of the bot, all of which has been submitted. The versions are "One-Base", "Always-Expand" and "Condition-Expand". The first version never expands, the second expands whenever desirable. The third behaves like the first version, until the opponent builds defensive structures where it switches to the second. Table \ref{tab:results} contains the results of the three versions.

\begin{table}
\begin{center}
\begin{tabularx}{\linewidth}{|X||l|l|l|}
	\hline
	Version				& Games	& Wins	& Win-rate	\\
	\hline
	One-Base			& $780$	& $527$	& $79.23\%$	\\
	Always-Expand		& $104$	& $78$	& $75.00\%$	\\
	Condition-Expand	& $79$	& $63$	& $79.75\%$	\\
	\hline
\end{tabularx}
\end{center}
\caption{Sparring results from the SSCAIT website}
\label{tab:results}
\end{table}

The initial One-Base version did very well, netting almost an $80\%$ win-rate. The next version, Always-Expand, was not tested as thoroughly, but it is clear that it did not perform as good. It turned out that by far most of the bots were aggressive rushers, which often would destroy the first expansion. At that point the bot would be behind in troops and quickly lose. This spurred the Condition-Expand version, where it would not expand against rushers. Unfortunately, it only marginally improved the win-rate, and with few enough games for the results not to be conclusive. This is probably because the expansions was not enough to win against the defensive bots, in which case the outcomes of games were unchanged.

As one could expect, the bot does well against others using boom-strategies and proxy-strategies. It wins by far most of the matches against other rush bots, including mirror-matches, except UAlberta bot as mentioned. It fares poorly against the turtling Terran bots, which quickly goes for more advanced units. The expansions were an attempt to combat the turtle strategy, but it was not enough. While the bot does gain an economic advantage, it never uses it to train stronger units.

SSCAIT accepts bots from non-students which only play in mixed-division where student bots play both that and the student-division. The statistics above is from the mixed-division, and there is unfortunately not a student-division only statistic. The One-Base and Condition-Expand versions are usually among the top 8 mixed-division bots and contender with a few for the first place in student-division. This is difficult to ascertain without clear statistics and will first be determined with the tournament results.

The bot will be entered in the 2015 AIIDE and CIG competitions, and further development on the bot is planned.

	\subsection*{Performance}
	Performance wise, the final version(s) of the bot do very well. SSCAIT monitors the frames as there are time limits bots must satisfy to upkeep the real-time aspect of the game. The rules are:
	\begin{itemize}
		\item no more than 1 frame longer than 10 seconds,
		\item no more than 10 frames longer than 1 second and
		\item no more than 320 frames longer than 55 milliseconds.
	\end{itemize}
	The bot has never lost a match due to breaking the time limits. In fact, it has not been observed to once exceed 55 milliseconds in a frame. Not unexpected however, since the bot has favored simple and very cheap solutions. There are certainly some algorithms that could be improved, such as detecting troops' proximity to the enemy. These are the minority however, and the current implementation allows for further development without reworking it.

	\subsection*{Features}
	The bot never reached any high-level, complex AI solutions. The focus was creating a competitive bot, so development focused on areas that improved performance the best. This turned out to be low-level technical features such as managing and commanding units and internal production data-structures. It was much more important that the bot acted correctly than it acted optimally, so simple solutions was sufficient in almost all aspects. Further development should focus on improving combat predictions and map awareness with regards to enemy unit placement, such that the bot can play in later stages of the match. Additional features would be controlling more kinds of units and better adaption to opponents' strategies.
	
	As the results show, the expansions did not improve performance noticeably. The economic advantage gained from expansions were not enough to offset the disadvantage of only using the earliest Protoss combat unit. By increasing the array of units used by the bot, the advantage of expansions may be clearer. In any case, expansions were needed at some point in the bot's development, as they are mandatory in longer games. In hindsight, it might have improved gameplay performance more if other features were pursued instead, like combat prediction or advanced units.