\chapter{Project Process}
The bot was developed with the agile development process. The implementation and design was iterated upon during short week-long sprints. After each there would be an evaluation of the bot's current status and new goals would be identified for the next sprint. A backlog kept during development was consulted when determining the status and goals in scope of the whole time-plan. There were a few set milestones throughout development, usually a month apart. They determined some requirements the bot should satisfy in relation to its performance at specific points in the development. The milestones served as a medium between sprints and the overall goal, ensuring the bot improved and did so consistently.

\section{Testing}
Testing was done in two steps. First the robustness of the implementations was tested by playing the bot with random built-in StarCraft opponents and tournament maps. This was to ensure the stability of the bot and that the implementations worked as required. The second step was uploading the bot to the SCCAIT tournament page, where the bot would play against the other official bots. This would serve to test the strategic integrity of the bot, and was done rarer than the first step. Because it is more difficult to retrieve data from rounds on the SCCAIT, the first step was needed to determine crash causes and such technicalities.

The opponents on the SSCAIT should be at least updated since that last tournament, and would therefore be viable candidates to test the bot's strength. It is required to make ones bot open source if it is to compete, so there is no reason to keep the newest bot version a secret beyond a year.

\section{Development Results}

\section{Evaluation}
The first milestone were attaining the minimum viable product. The requirements for this was a bot capable of defeating a passive opponent, implying means to gather resources, build units and attacking. This would imply the bot had a non-zero chance at winning a round, making it an actual competitor.

%When developing the bot, BWAPI documentation and the StarCraft Broodwar AI forum was often consulted on technicalities, while competitive StarCraft communities and wikis were used when designing strategies. Both StarCraft and its bot development is very community driven with lots of resource available.

%Timeline?
%Milestones

