\chapter{Project Process}
The project was developed with the agile development process, but no particular agile method was used. Design and implementation was iterated upon across short week-long sprints. After each there was an evaluation of the agent's current status and new goals would be identified for the next sprint. The project plan took form as a backlog, kept during development, and was used when determining the status and goals in scope of the whole time-plan. There were a few milestones throughout development, usually a month apart. They determined some requirements the agent should satisfy at specific points in the development, with regards to its performance. The milestones served as a medium between sprints and the overall goal, ensuring the bot improved consistently, and that a satisfactory end-result would be met.

Testing was done in two steps. First the robustness of the implementations was tested by playing the bot with random built-in StarCraft bots, across all maps used in the competitions. This was to ensure the stability of the bot and that the implementations worked as required, across all the maps it would play on. The second step was uploading the bot to the SCCAIT tournament page, where the bot would play against other bots. This would serve to test the strategic integrity of the bot, and was done rarer than the first step in order to retrieve enough statistical data. This was much slower as the matches were tested in real-time, but necessary to ensure the opponents were up-to-date. The opponents on the SSCAIT website should be as recent since the last tournament (January, 2015), and would therefore be the best candidates to test against the agent's abilities. Because it is time-consuming to retrieve data from rounds on the SCCAIT, the first step was needed to determine crash issues and such technicalities, in order to not waste time.

\section{Development Results}

\section{Evaluation}
The first milestone were attaining the minimum viable product. The requirements for this was a bot capable of defeating a passive opponent, implying means to gather resources, build units and attacking. This would imply the bot had a non-zero chance at winning a round, making it an actual competitor.

%When developing the bot, BWAPI documentation and the StarCraft Broodwar AI forum was often consulted on technicalities, while competitive StarCraft communities and wikis were used when designing strategies. Both StarCraft and its bot development is very community driven with lots of resource available.

%Timeline?
%Milestones

