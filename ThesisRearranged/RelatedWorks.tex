\chapter{Related Works}
Bots in StarCraft

\section{Strategies in StarCraft}
A very important part of StarCraft strategies are the openings. Much like chess openings, StarCraft has a wide array of detailed steps to take in the first few minutes of a round.

In most RTS games, StarCraft being no exception, the trinity of strategies are \emph{rush}, \emph{boom} and \emph{turtle}.

The \emph{rush} strategy is when the player attempts to produce and attack with troops as fast as possible. There are many variations in StarCraft, where the most aggressive are called \emph{all-in}, as the rushing player will lose shortly afterwards if the initial rush fails. This could happen if the opponent builds stationary defenses, which are usually much more cost-effective than the mobile units required to attack.

Closely related, a player could instead focus on developing economy or technology to gain advantage against slow strategies. This is usually called a \emph{boom} strategy, which in StarCraft involves gaining an \emph{economic advantage}. If successful, the player will attain superior units and easily replace lost ones, winning either through strength or attrition. The weakness is early game defense will have to be sacrificed which puts the boom player at risk to enemy rush strategies.

Finally, the defensive strategy called a \emph{turtle} or \emph{turtling} builds strong defenses early on. As mentioned, this will counter an enemy rush by costing fewer resources, used to expand in economy or technology. The resulting advantage is then used to win. This is a slow strategy however, which fails against the riskier \emph{boom} strategy. Since the defensive strategy does not put pressure on the opponent, they are free to completely spend resources on expanding to stronger units.

Usually either of these strategies are used during the early game. Later on, players will have more resources to pursue multiple goals and a more balanced overall strategy. This takes form in StarCraft by players attempting to gain a long-term economic advantage, done by controlling the limited resource locations. While early game can be decided by a single battle, later stages require multiple.

\section{Bots in StarCraft}
Usually bots focus on only one strategy. Some of the more advanced bots identify whatever opening strategy the opponent is using.

Few bots reach the late-game content of StarCraft. Those that do usually focus on the incredible precision and speed which the bot can play compared to a human.

%ICEbot, Krasi0, tscmoo + tscmooz, Martin Rooijackers, UAlberta, Black White, Tomas Vajda, Andrew Smith