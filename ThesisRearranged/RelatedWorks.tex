\chapter{Bots in StarCraft}
\label{ch:related}
Before delving into the agent's design, this chapter will cover a few other bots, their solutions and the competitions in which they play. First we will explain the basics of strategies in StarCraft.

In most RTS games, StarCraft being no exception, there is a trinity of strategies in a rock-paper-scissor formation. These are the \emph{rush}, \emph{boom} and \emph{turtle} strategies.

The \emph{rush} strategy is when the player attempts to produce and attack with troops as early as possible. There are many variations in StarCraft, where the most aggressive are called \emph{all-in}, as the rushing player is bound to lose shortly afterwards if the initial rush fails. This could happen if the opponent builds stationary defenses, which are usually much more cost-effective than the mobile units required to attack.

Closely related, a player could instead focus on developing economy or technology to gain advantage against slow strategies. This is usually called a \emph{boom} strategy, which in StarCraft involves gaining an \emph{economic advantage}. If successful, the player will attain superior units and easily replace lost ones, winning either through strength or attrition. The weakness is early game defense will have to be sacrificed which puts the boom player at risk to enemy rush strategies.

Finally, the defensive strategy called a \emph{turtle} or \emph{turtling} builds strong defenses early on. As mentioned, this will counter an enemy rush by costing fewer resources, used to expand in economy or technology. The resulting advantage is then used to win. This is a slow strategy however, which fails against the riskier \emph{boom} strategy. Since the defensive strategy does not put pressure on the opponent, they are free to completely spend resources on expanding to stronger units.

Matches in most RTS games can be divided into three stages: \emph{early-}, \emph{mid-} and \emph{late-game}. These are somewhat vague definitions of time intervals in the game where different strategies apply. There is no match timer in StarCraft, so the stages are usually determined by the game-state, and therefore they arrive at different times across matches. Mid-game is when players have established their base and economy and usually built more bases. Late-game is once most of the map has been settled and all units are available to players. The three basic strategies we covered are used during the early game, but have strong influence on mid- and late-game.

\section{Bots}
Computer-controlled players in games are called \emph{bots}. The main difference between agent and bot is that bots are always a replacement for players, and they are usually limited by the same rules as human players. The term is mostly used in first-person shooters, but can be extended to all video games. Both bot and agent is used in this paper to refer to AI-controlled StarCraft players.

Compared to other games however, even the best bots in RTS games are usually worse than competitive human players. This is a result of the difficult challenges bots must overcome in the games, particularly the huge decision space. At every frame, a player is capable of giving one or all of their units new commands, most of which require a target unit or location. Couple a 55 ms. limit in the SSCAIT and similar in other tournaments, bots do not have very long to perform calculations on this decision space. While the same issue lies in Chess turns, there are many more frames in a typical StarCraft match, which is usually among the ten-thousands.

It is practically not possible to use any standard AI solution that scales with the decision space, like alpha-beta pruning, without massive simplifications. Bots usually resort to having a few, very rigid move-sets, with some hard coded adaption to usual enemy strategies.

Few bots ever reach mid-game or later as matches are often determined beforehand. As the game progresses to later stages, the races gain access to many kinds of units, which are difficult to control well for a bot. Most bots therefore focus on the early-game where only a few units are available, unable to perform well at later stages. The most advanced bots, capable of playing the later stages well, forces the match into mid-game by turtling, winning against the simpler bots. Notable examples of this strategy are Letabot and XIMP bot.

The following sections outline some of the challenges in StarCraft bot development and some solutions.

	\subsection*{Incomplete Information}
	As the map is partially shrouded in the fog-of-war, players are left to guess what the opponents' are scheming. While humans are pretty good at this, it becomes troublesome when bots have to model the movement of opponents. The longer they are left unchecked, the more possible game-states are there to account for. Scouting opponents is therefore imperative, as this will reduce the amount of prediction needed.
	
	Beyond the fog-of-war, upgrades are impossible to discover unless their effects are observed. Experienced players can often guess the state of opponents' research based on their strategy, and in some cases have memorized some unit match-ups upgraded and otherwise. The accuracy of combat predictions rely on this information, which are already difficult to do accurate without factoring in upgrades. Opponents' resources are also hidden from players.
	
	%http://agents.fel.cvut.cz/~certicky/files/publications/tciaig2014.pdf
	%http://skemman.is/stream/get/1946/10688/26112/1/MSc.pdf Bayesian network to represent technology
	
	\subsection*{Controlling Multiple Units}
	There is a lot more to RTS combat than commanding unit A to attack B. Positioning alone is a huge factor, which is completely dependent on the positioning of all other units in the area. The terrain of the map can be advantageous or otherwise. There are different ways to prioritize targets, none of which are conclusively superior than others. Humans have the advantage by easily reading the graphical display of RTS games, where the bot must use other methods.
	
	One of the solutions practiced by ICEbot is using potential flows to direct units' positioning. A similar solution is also used when scouting in order to avoid enemy troops. Potential fields is used by the Berkeley Overmind for guiding its air-units around threats, and heatmaps have been used to detect army compositions. \cite{Overmind} \cite{Potential12} \cite{PotentialNav}

	\subsection*{Strategic Planning}
	To create more adaptable bots, there has been several attempts to use some dynamic decision making solutions. Once the opponents' game-state is known, it is possible to infer their strategy and future actions, which can be used to react and prepare a counter-strategy.
	
	There has been multiple studies about using Bayesian models for planning and prediction. \cite{BayTac}
	%http://link.springer.com/chapter/10.1007/978-3-642-37798-3_38
	%http://emotion.inrialpes.fr/people/synnaeve/index_files/SpecialTactics.pdf
	%http://emotion.inrialpes.fr/people/synnaeve/index_files/OpeningPrediction_proceedings.pdf
	%http://emotion.inrialpes.fr/people/synnaeve/index_files/AIIDE_11_RC1.pdf
	%http://emotion.inrialpes.fr/people/synnaeve/index_files/BayesianUnit.pdf
	%https://hal.archives-ouvertes.fr/hal-00752868/
	
	%http://agents.fel.cvut.cz/~certicky/files/publications/tciaig2014.pdf