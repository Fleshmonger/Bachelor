\chapter{Bots in StarCraft}
Before delving into the agent's design, this chapter will cover a few other bots, their solutions to the AI challenges and the competitions in which they play. First we will explain the basics of strategies in StarCraft.

In most RTS games, StarCraft being no exception, there is a trinity of strategies in a rock-paper-scissor formation. These are the \emph{rush}, \emph{boom} and \emph{turtle} strategies.

The \emph{rush} strategy is when the player attempts to produce and attack with troops as fast as possible. There are many variations in StarCraft, where the most aggressive are called \emph{all-in}, as the rushing player will lose shortly afterwards if the initial rush fails. This could happen if the opponent builds stationary defenses, which are usually much more cost-effective than the mobile units required to attack.

Closely related, a player could instead focus on developing economy or technology to gain advantage against slow strategies. This is usually called a \emph{boom} strategy, which in StarCraft involves gaining an \emph{economic advantage}. If successful, the player will attain superior units and easily replace lost ones, winning either through strength or attrition. The weakness is early game defense will have to be sacrificed which puts the boom player at risk to enemy rush strategies.

Finally, the defensive strategy called a \emph{turtle} or \emph{turtling} builds strong defenses early on. As mentioned, this will counter an enemy rush by costing fewer resources, used to expand in economy or technology. The resulting advantage is then used to win. This is a slow strategy however, which fails against the riskier \emph{boom} strategy. Since the defensive strategy does not put pressure on the opponent, they are free to completely spend resources on expanding to stronger units.

Matches in most RTS games can be divided into three stages: \emph{early-}, \emph{mid-} and \emph{late-game}. These are somewhat vague definitions of time intervals in the game where different strategies apply. There is no match timer in StarCraft, so the stages are usually determined by the game-state, and therefore they arrive at different times across matches. Mid-game is when players have established their base and economy and usually built more bases. Late-game is once most of the map has been settled and all units are available to players. The three basic strategies we covered are used during the early game, but has strong influence on mid- and late-game.

\section{Bots}
Few bots ever reach mid-game or later as matches are often determined beforehand. As the game progresses to later stages, the races gain access to many kinds of units, which are difficult to control well for a bot. Most bots therefore focus on the early-game where only a few units are available, unable to perform well at later stages. The most advanced bots, capable of playing the later stages well, forces the match into mid-game by turtling, winning against the early bots.

Usually bots focus on only one strategy. Some of the more advanced bots identify whatever opening strategy the opponent is using.

Few bots reach the late-game content of StarCraft. Those that do usually focus on the incredible precision and speed which the bot can play compared to a human.

%ICEbot, Krasi0, tscmoo + tscmooz, Martin Rooijackers, UAlberta, Black White, Tomas Vajda, Andrew Smith


\section{AI Competitions}
AIIDE, CIG, SSCAIT.
Different competitions and their descriptions.

Competition rules: Setup, limits, goals.

Ties impossible

This bots involvement/predicted involvement.