\chapter{Exploration}
\label{ch:exploration}
One of the challenges in StarCraft is the imperfect information players have of the world. It becomes necessary to deduce the opponent's strategy from a few units, and regular scouting is mandatory. The player with the best information coverage can make the most optimal decisions. This chapter describes how the agent solves the information and scouting problems.

All units, including buildings, have a line of sight equal to their \emph{sight} value. Within this distance, the map is revealed to the players. Everywhere else, only the terrain features are visible or buildings in their last seen state. This is the \emph{fog of war}, and all units within this are \emph{invisible}. These units cannot be targeted or attacked, and the agent cannot access any data of them through BWAPI. It is up to internal data systems to store this information.

The entire map terrain, including initial resources, are revealed to players. All possible start locations are also known, but it is not known which the opponents occupy.

Ground units cannot see past cliffs or ramps, giving an edge to units on the high ground. Most start locations are located on plateaus, making them easily defended. Units or obstacles do not break line of sight however. Some units are \emph{cloaked} or can \emph{burrow} which renders them invisible to normal units even outside fog of war. BWAPI however does not limit the agent from accessing their data while they are within line of sight. Units with the \emph{detector} ability reveal cloaked units within its line of sight.

The agent must track data about discovered units, which is handled by the \emph{Archivist} and \emph{Geologist} module, the subject of the first section. The second describes the scout AI, implemented by the \emph{Reconnoiter} module.

\section{Tracking Units}
Two things must be recorded for each enemy unit: position and type. The former is needed to track movements of the enemy army and to record enemy base locations. The latter is needed since the unit type contains all relevant values such as speed, damage and maximum health. Some units can change their type even, requiring regular updates of changes. The Siege Tank for example changes types when it goes into or out of siege mode.

The \emph{Archivist} handles all known enemy units. The solution is sets of units and dictionaries of their data, where units are keys to their values. Every frame, the Archivist checks all stored units, updating their values if they are visible. Whenever a unit is first discovered or destroyed, BWAPI will notify the Archivist, which will then modify the dictionaries and sets appropriately. This solution spends logarithmic time on queries, insertions and deletions. The position and type queries could be improved to amortized constant time with hashing.

Units are inserted into multiple sets, separated in categories. Some units are stored in more than one set such that constant time retrieval is possible, at the cost of more marginally more space. These sets are used by other modules when specific kinds of units are desired, such as structures or workers.

The Archivist is one of the lowest in the hierarchy, as it depends on no other module and is used by almost all others. Other modules query the archivist for the units or their values.

	\subsection*{Resource Locations}
	As all initial resources and their positions are retrievable from BWAPI at any time. Since new minerals are never created, it never becomes necessary to keep track of them here, as only the Gatherer module from chapter \ref{ch:resources} needs them.
	
	Vespene geysers however have special behavior in StarCraft, as mentioned in chapter \ref{ch:production}. When a refinery is built upon a geyser, it is morphed rather than constructed. However, if the refinery is destroyed, a new, distinct geyser unit is created in its place. While all the initial geysers can be retrieved from BWAPI, new ones cannot. Otherwise, agents could deduce if any opponents lose refineries. This makes the tracking of geysers a bit more tedious than other units.
	
	The \emph{Geologist} keeps track of current known geysers in the map. They are stored in a dictionary of sets, where the keys are regions and the sets contain geysers from that region. This is because superior modules need the geysers separated in regions. The Geologist responds only to events, inserting new geysers or deleting refineries when discovery. All initial geysers are added on initialization. Since a refinery were the same unit as the late geyser, they can be used to remove them from the sets. Given $r$ regions and $g$ geysers, insertions and deletions cost $O(log r + log g)$ time. Since every region rarely has more than one geyser, the time is more like $O(log r)$, which is the cost of a query. As usual, this could be improved with hashing to amortized constant time.

\section{Scouting}
Scouting is important throughout the entire duration of the game. Early on, players need to know where the opponent is and which opening they are using. Later on, observing the opponent's army size, unit types, base expansions and tech level is important to properly combat their strategies. The opponent's race is known unless they pick random, in which case this also needs to be scouted. There are no bots playing a random race however, so this is not relevant to us.

Recall that all possible player start locations are revealed to players. Scouting for the opponent base is then just visiting other start locations than ones own. Maps used in the competitive bot matches contain no more than four start locations. It could therefore be the case that a player has to scout up to two bases before knowing the enemy start location. Usually a player wants to know more than where the opponent starts, so even when it is deductible where the opponent is, the player wants to scout the base itself.

Once the enemy base has been revealed, and with it his opening strategy, the scout is not useful any longer. It is a long trip back to gathering resources and the scout could be followed, revealing your own location. Harassing enemy workers however can disrupt the opponent's economy, even if none of them are destroyed. Attacking enemy workers forces the opponent to pull at least one worker from gathering to defending. Usually there are no base defenses or troops at this early in the game, except if the scout had to visit three bases. Scout harassment can be very advanced, as the scout can retreat while chased, possibly attacking passive workers. Killing even one worker is a big advantage this early in the game.

The \emph{Reconnoiter} module contains the scouting AI. Using the TaskMaster module from chapter \ref{ch:resources}, the Reconnoiter picks some free worker as the scout. As usual, a worker is considered free if it is either idle or gathering resources. The scout is removed from the worker pool, as it will be a scout for the rest of its life span.

Usually a Protoss player will send the scout after the first pylon has been built, where the builder is used as a scout. As an approximation, the Reconnoiter will only acquire a scout if a specific supply total of $8$ has been reached. This is because the first pylon is usually built around eight workers, such that the Reconnoiter should acquire a scout at the same time the pylon is built. The worker will never be picked however, as it is not considered free the moment the supply limit is reached.

If the Archivist has not recorded any enemy buildings and the Reconnoiter has no scout, it will attempt to acquire a scout. This implies scouting will proceed as early as possible and if all known enemy buildings has been destroyed, extending its use into later stages of the game.

While the Reconnoiter has a scout, it picks an unexplored base location and moves the scout towards there. A tile is considered unexplored  by BWAPI if it has never been revealed for the duration of the game. Note that this removes the agents own starting location from the candidates. Once the scout reaches the destination the tile will be explored, removing it from the possible scouting locations. In case this is the opponent's starting location, their depot will be revealed as well.

There are alternative uses of the scout compared to harassing, such as building blocking the opponent's main base exit with defensive structures or building troop producers close by. These however must be planned in concordance to the grand strategy, as they involve spending resources. Some bots use these strategies, LetaBot for example can rush with bunkers. %TODO Reference and more examples.

At later stages in the game, faster units should be used to scout the map. In particular the Protoss Observer is a good scouting unit, as it is both permanently cloaked and a detector.