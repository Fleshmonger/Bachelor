\chapter{Combat}
\label{ch:combat}
This chapter explains how combat is handled in StarCraft, how its outcomes can be predicted and how the agent manages troops, attacking and defending.

Combat in StarCraft can easily become the most complex part of bot development. Multiple studies has been done on just small subsets of artificial combat scenarios, with specified units and simple terrain, but even these just scratch the surface of the field. Like the macroscopic strategies in StarCraft, there is likely no optimal command scheme for ones units. Predicting the outcome of a combat scenario, is therefore only done as approximations. This is also one of the areas where bots can excel against humans, since they can easily command different units in complex ways.

\emph{Troops} or \emph{fighters} here means units used for combat. Almost all non-building units beyond the workers are used in combat. Some units called \emph{spellcasters} deal damage indirectly through their abilities, while \emph{support} units help others fight better.

Every unit has some amount of \emph{health}, and almost all units can \emph{attack} to deal \emph{damage} to others. There are a number of additional factors such as armor and damage types, splash and hit probability, but it is sufficient to know that a unit loses health relative to the damage it is dealt. When a unit reaches 0 health, they are destroyed. Units attack at different ranges, such that \emph{melee} units need to be close and \emph{ranged} units can attack at a distance.

In the following sections we will explain the base management of troops. Then we move on to combat prediction, followed by attacking and defending.

\section{Managing Troops}
The \emph{Army Manager} is the troop parallel to the Task Master from Chapter \ref{ch:resources}. It contains all fighter units controlled by the agent and their current assigned \emph{duty}, same as the \emph{tasks} for workers from Section \ref{sec:manageWorkers}. The current types of duty are \texttt{idle}, \texttt{defend}, \texttt{attackTransit} and \texttt{attackFight}. Distinct from the Task Master, there is only one instance of the Army Manager, as troops are not separated into regions.

\section{Combat Prediction}
Predicting the victor of combat is very difficult in StarCraft and predicting the specific casualties is neigh impossible. Even beyond the numerous factors in combat, the optimal command of troops is completely Dependant on the opponent's commands. Unless the opponent operates in a recognizable pattern, they movement is unpredictable. Predicting combat is therefore more based on what units the opponent controls, how the terrain is and then either a theoretical upper limit of their damage output versus a prediction of our own.

Some of the most successful predictors simulates a simple form of combat and evaluates the result. This however assumes how the opponent will control its troops, or at least assumes it knows the optimal control scheme. Neither of these are possible, especially not with incomplete map information, but the results are usually close enough. SparCraft uses this method with good results, but even it ignores multiple factors including collision.

The combat prediction algorithm resides in the \emph{Combat Judge} module, which is used only by the Attacker and Defender, covered in this chapter, and the \emph{Strategist} covered in Chapter \ref{ch:strategy}. Given a set of units, it returns a value representing the \emph{strength} of them. The army with the higher value is predicted to win in a confrontation.

Following is a description of the prediction heuristic: let us assume a simple scenario: two units, $a$ and $b$ are attacking each other. We have the health values $h_a, h_b$ and the \emph{damage-per-second} (DPS) values $dps_a, dps_b$. Assuming both units start attacking each other at the same time, then we must have $TTK_a = h_b / dps_a, TTK_b = h_a / dps_b$, where $TTK_a$ and $TTK_b$ are each unit's \emph{time-to-kill} (TTK). This value represents the time in seconds it takes to kill the opponent. It must be the case that the unit with the lowest TTK becomes the victor, and with this we can predict the outcome of the simple situation.

Notice that we can simplify the solution and instead simply compare the values $s_a = h_a dps_a$ and $s_b = h_b dps_b$. This enables us to value the strength of an unit independently of the enemy, which is quite helpful when pulling workers one at a time in Section \ref{sec:defending}.

We would like to scale the solution to armies by using the sum of strengths as the army's strength. This requires the additional assumption that no units are destroyed during combat. If this happened the TTK would not be constant, changing the outcome. This is almost never the case in real combat scenarios, and can cause the prediction to be inaccurate. This happens when armies have the same strength value, but not the same amount of troops, where the largest army is overestimated.

The prediction was already inaccurate however, as we never take terrain or range into account. For this reason, the prediction underestimates Terran units, which are all ranged. Additionally the prediction has two properties that are not true. Firstly the strength of armies increases linearly with its size, which is not necessarily true in-game. When grouped, the units destroyed last will have dealt more damage than possible if attacking one at a time. Secondly the comparison between armies is non-transitive. The units in StarCraft (and most RTS games) are designed like a rock-paper-scissor game, such that army $A$ beats $B$ and loses to $C$ while $B$ beats $C$. This means the strength of an army is relative, and cannot be accurately evaluated as an absolute value like here. 

However, this solution has proven to be sufficient in the early game. Protoss has the strongest early unit, which mitigates the disadvantage of attacking when outnumbered. The units will usually be able to destroy one or more of the enemies, such that both players loses troops. The only case where this is not true is against other Protoss players, but here the prediction is perfect as both use the same early unit type.

The prediction is very cheap, having a time complexity linear to the size of the army and uses constant amount of space.

\section{Attacking}
\label{sec:attacking}
A player will almost certainly expand to their natural expansion first and expanding to a third base is only viable beyond early game. The natural expansion is very close to the main base and is usually right outside the only exit of the main base. This proximity allows us to consider both bases as a single base. From these assumptions we can conclude the opponent will only have one base for the duration of the early game, which is the only stage we are concerned with.

In euclidean space the shortest path between two points is the same regardless of which is the origin. In terms of StarCraft, the shortest path between the opponent's and the agent's base is the same regardless of which the troops move from. Assuming each player only has one base and both players will use optimal pathfinders, we can model the map as a straight line between the bases. Both armies will only travel along the shortest path and there is therefore no case of armies moving past each other without colliding.

When rushing, it is favorable to move troops as close to the opponent's base as possible without confrontation, even before they are ready to attack. This pressures the opponent, scouts the map, attacks the earliest possible time and distances the opponent's troops far from the player's base. Even if the fight ends in defeat, the opponent's troops will have to move across the map to attack, and in case of victory, the player's troops can continue towards the opponent's base with minimum delay.

When the we predict a victory in combat against the opponent army, we attack. In some cases it might be prudent to wait until more units have been amassed, especially if we produce more troops than the opponent, as we will sustain fewer losses and be more certain of a victory. This is difficult to asses while also being very risky, so the safer option is just attacking immediately.

Attacking is handled solely by the \emph{Attacker} module.

	\subsection*{Targeting}
	Most real-time strategy games has the unit command \emph{attack-move}, and StarCraft is no exception. A unit executing this will move towards a unit or location, but will attack any enemies along the way. The unit will prioritize attacking those that are damaging it. This command therefore auto-targets nearby enemies and also has simple, inherent prioritized targeting.
	
	By targeting the enemy base location or structures with this command, troops will automatically fight the first enemy they meet, which is either the opponent's base or army. The most important targets are the enemy structures, since destroying all of them is the win condition. The Archivist from Chapter \ref{ch:exploration} records all known enemy buildings, from which the Attacker picks one arbitrarily. With the assumption that the opponent only has one base, the army will attack the foremost building regardless of target, since they are attack-moving. In case there are defensive structures, the attack-move command will automatically re-target to them once they damage the army.
	
	This allows for constant time targeting retrieval. Alternatively some units could be prioritized, such as the enemy workers or troop producers, but this is otherwise the cheapest solution while still a good approximation of optimality.
	
	Some bots manually prioritize during combat. There have been studies on using different priorities, such as attacking the unit with highest health to resource cost ratio. This is a greedy attempt at costing the opponent as much as possible. A similar one is attacking units with highest health to damage ratio, or simply the unit with lowest health. There are more advanced strategies however when considering large scale armies composed of different unit types, where counter-play between individual types are important. The usual order of targeting is the opponent's army, then economy and then unit producers.
	
	\subsection*{Troop Rendezvous}
	Once any fighter unit has been completed, they are set to the duty \texttt{idle}. Every frame the Attacker will re-task all the \texttt{idle} troops to \texttt{attackTransit}. These units are commanded to move towards the enemy base, but not attack. Otherwise, they would attack the enemy base one at a time. Grouping troops is more efficient as the total health pool is larger, allowing the troops to deal more damage before they are destroyed. This has the opposite effect on the opponent's army, which has less time to deal damage. Therefore, it is more efficient to wait until more troops have arrived.
		
	At some point the troops will be within some distance of enemy units where they will risk being attacked. This distance is a bit larger than the longest range of any unit. Once troops are within this distance, the Attacker will use them for combat prediction. By only counting these arrived troops, the module can tell when there are enough to beat the opponent. If they are strong enough, the Attacker will re-task them to \texttt{attackFight} duty. In this state, they are commanded to attack move the target found prior. It is important to attack immediately when possible, as waiting too long will allow the opponent to counter the attack with stronger units or defenses.
	
	In case the \texttt{attackTransit} units within this danger distance is not strong enough, they will instead retreat towards the main depot. The result is troops moving back and forth right outside of enemy range, until enough troops have gathered. In case the opponent advances with a stronger army, the troops will retreat until outside their range. In case the enemy army reaches the base, the Defender module takes over the troops, which is explained in Section \ref{sec:defending}.
	
	\texttt{attackFight} units never retreat, fighting to the death. This is because it is difficult to determine when retreating is efficient. While troops retreat, the enemy army has the opportunity to attack them without receiving any damage themselves. In some cases retreating might be impossible if ones units are too slow. In case the army would sustain heavy losses, it might have been a better choice to destroy as many of the enemy troops as possible. All troops with the \texttt{attackFight} duty outside of enemy range reverts to the \texttt{attackTransit} duty, moving towards the enemy base once again.
	
	The Attacker only counts units currently within this danger distance of an enemy unit as arrived, that is, it does not record arrivals between frames. Testing proved the battlefield shifted too much to record arrivals and stopping troops until they were ready, as it lead them to becoming scattered across the battlefield sometimes. By only relying on the current situation, it is certain that the units are actually gathered.
	
	This current implementation is expensive as every unit compares itself with every known enemy to detect whether they are within range. Given $n$ troops and $m$ enemies, the time complexity is a polynomial $nm$ number of operations. All the remaining operations are asymptotically less than this however, so this is also the total. 
	
	There is not always enough space for all units to wait if the army is large, but it is usually not. A better solution would be needed however if the army was larger or contained large units. In this case, heat maps or cluster detection could be used to detect the density of the army.
	
\section{Defending}
\label{sec:defending}
If a player's base is under attack, different behavior is required compared to attacking. Even if outnumbered, it is favorable to fight the invaders. In case the defending player has no troops left, the workers function as a last line of defense. Usually the opponent will attack the workers first in any case, so they might as well fight back.

Since Protoss are quite slow, it is possible the opponent rushes first. In some cases, no troops might have been built yet, and it is up to the workers to defend the base. Zerg is very quick to produce troops and is capable of this, but other races could send workers to build a forward base. The usual case however is defending against the opponent's scout. Even if the scout simply stays in the base without attacking, it is important to destroy it quickly to starve the opponent of information.

Some strategies involve building static defense structures, which are stronger and cheaper than the mobile troops. These can be built very early if the player is determined, at the cost of expanding economy and not pressuring the opponent. Beyond early game, these defensive structures are usually always built to prevent ambushes and detect cloaked units. Rushing players however, favors mobile troops over these defenses.

The \emph{Defender} module is capable of defending with nearby troops, including workers if necessary, against any non-flying, non-cloaked threats. The module is a sibling to the Attacker in the hierarchy, but has a higher priority as defending the base is more important than attacking. Therefore the Defender is updated before the Attacker in each frame.

	\subsection*{Scrambling Defenders}
	The problem with acquiring defenders is related to the retreating problem. It must be avoided pulling troops back that would be better used attacking. Defending is an attempt at damage control, however the opponent must still be pressured with attacks as quick as possible. Some units might be too far away to participate significantly in the defense, in which case it should instead attack the opponent. In case both players are invading each others' bases, it becomes a race. In these situations, it is not always clear what the correct move is.
	
	In line with the non-retreating paradigm from Section \ref{sec:attacking}, we will assume that units already in combat should not be interrupted. Additionally, units within regions with expansions should defend, as the unit will be within reasonable traveling distance and is not currently attacking. Since the macro strategy is rush, units that are not within an region with expansions should continue to attack.
	
	As it turns out from testing and because of the single path model, the agent very rarely have any units left outside its base if the opponent is currently invading, as all prior units would already have fought and been destroyed. In case there are units beyond, they will usually have reached the opponent's base before his army invaded, in which case the correct decision is continuing the assault. The current solution has therefore proved to be sufficient.
	
	There is a slight issue because of the region clause. If the opponent has a large army which has not completely crossed into the defended region, the defenders will revert into attacking behavior upon destroying all those within the region. Sometimes the defending units will continually retreat and attack as parts of the enemy army moves into the region, sustaining more damage than is necessary. It is a difficult problem to solve however, as the Defender would need to detect and target the clusters of units rather than the individuals.

	Once all defenders have been assigned, they are commanded to attack-move an arbitrary invader. The defending units will then prioritize the closest invaders. If the invaders are defeated, all the remaining defenders will be set as \texttt{idle} in the Army Manager.

	\subsection*{Rallying Militia}
	Sometimes there are not enough troops in the base to defend. In this case, a common strategy is moving all the workers to defend.
	
	Once all available defenders have been assigned, the Combat Judge will calculate the values of the invaders and defenders. Until there are no more free workers or the defenders are stronger than the invaders, the Defender will task additional workers to defend. They are commanded in the same fashion as the rest of the defenders. Once the fight is over, all the defending workers will be re-tasked as \texttt{idle} in the Task Master.

	Alternatively, in case there are enough troops to defend, players might temporarily relocate workers to safe resource clusters until the conflict is resolved. This is rarely an issue in the early game however, and has not been a priority yet.