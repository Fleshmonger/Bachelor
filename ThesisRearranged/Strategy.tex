\chapter{Macro Strategy}
\label{ch:strategy}
Two strategy terms within StarCraft is \emph{micro} and \emph{macro}. The former is a players ability to micromanage units and their commands, where the latter is general stratagem and economic management. Human players have to balance the two, but bots are not so limited. Up until now we have almost exclusively covered micro behavior, such as how units are commanded, produced and tracked. This chapter concerns the bot's macro strategy and its architecture, which covers economy, strategies and production.

The first section describes the strategy module hierarchy with \emph{Despot} as root module. The second section explains the opening strategies and how they are implemented in the \emph{Planner}. The third section details the economic decisions dealt by the \emph{Economist} where the fourth covers combat strategy handled by the \emph{Strategist}. 

\section{Strategy Architecture}
While the Architect and Recruiter handles the technical aspects of production, they never produce anything of their own volition. This is the concerns of the \emph{Planner}, \emph{Economist}, \emph{Strategist}. The Planner executes build-orders if any is queued. The Economist commands worker and supply production, where the Strategist handles troop and their trainers' production. All three are direct subordinates to the \emph{Despot}, which controls which of them are active. In case the Planner has a build-order queued, only it will be active every frame. Otherwise, the Economist and Strategist will be active while the Planner is not.

\section{Build-order}
\label{sec:buildOrder}
An \emph{opening} refers to the beginning strategy a player uses, for example rushing. Opening strategies usually include a \emph{build-order}, which namely is an order of units to build. It is a strict progression of the early game that has been found through experience, exactly like chess openings. They are designed to work against current strategies in the environment, but do rely on the player's judgment on whether to change opening.

Every unit in the build-order is set to be produced when a specific amount of supply has been used. It is implied that workers are produced until the next supply threshold is reached, upon which the next item in the order is produced. Some build-orders are more detailed, specifying placement or techniques to be used.

Because of the dependency on the environment, like all kinds of strategies, they are known to change every now and then as different strategies become popular between players. This is also apparent in the bot scene, where the over-saturation of one strategy encourages counter strategies. 

Build-orders are easily used by bots as they are a simple queue of orders. The \emph{Planner} has such a queue, which is executed at the Despot's discretion. If the queue is not empty, as many of the next items are executed instead of using the Economist or Strategist. They become used again once the queue is empty. This allows for constant time execution per item, barring the time to produce the items. Thus the implementation is very cheap as every item is only produced once and spread across many frames.

Instead of implying the production of workers between orders, the build-orders are modified to have them expressed as items in the order.

The benefit using build-orders with a bot is easily adding multiple openings for the bot, depending on the match-up. Although the general strategy remains, different openings can vastly change the outcome of the game. Proper choice of opening is key to winning in early game, or at least not lose during. The downside however is the rigid planning structure, which can't allow for easy adaptations or responses. In case some units are destroyed, it is possible that the current build-order is no longer optimal, viable even. Humans can easily adapt to these changes, but this is not always so with a bot, where canceling the build-order can be necessary.

We avoid this by only using short build-orders, such that the opponent could in no way interrupt with any units other than harassing workers. Even shortened, they are still useful.

This solution is also used to prioritize production. Since the Economist and Strategist use a greedy strategy, it can be difficult to build high-cost units such as the depots. By enqueue the depot, all other production will be paused until it can be created. This solution is risky however, as the bot will not consider removing the depot from the queue under any circumstance. Practically, it does not seem to cause problems, as the bot will reach the required amount of resources quickly after ensuing, resuming normal behavior again.

\section{Economy Strategy}
Economy includes producing workers, refineries and depots and managing these. The \emph{Economist} handles these aspects of the bot, updated at the discretion of the Despot.

	\subsection*{Resource Gatherers}
	Recall from Chapter \ref{ch:resources} that there is a specific maximum gathering workers on each resource. Any beyond this will not yield any returns. Therefore, the optimal worker amount is having the maximum allowed workers at each resource cluster.
	
	Generally it is advised that a Protoss player keeps producing workers constantly until late-game, but some openings might temporarily stop worker production. This is not the case here however. Therefore, while there are not the desired amount of workers in a base, we attempt to produce a new worker at that base. This is verified checked every frame. The asymptotic time-complexity is therefore linear to the amount of bases and the time required to produce a worker (see Section \ref{sec:training}). However we expect to only have one to three bases and the worker production is constant time since the depot is given as trainer.
	
	\subsection*{Expanding}
	\label{sec:expanding}
	Recall the \emph{expansion} from Section \ref{sec:buildExpansions}. Determining whether when to expand is difficult but pivotal to the match's outcome. It takes time and resources to build a new depot where it will be vulnerable before completion. A player with a stronger army can safely expand while guarding it, securing his current advantage or at keeping up with the opponents' expansions. Without the stronger army it is risky, but could become necessary. The resources in current bases will be depleted at some point, leading to a constant need to expand. Expanding before current resources are completely saturated with workers is a \emph{fast expand}, sometimes used as an opening. This is very risky, but with an immediate payoff in economic advantage. The contrary formula, keeping a single base until mid-game or alike is called \emph{one base play} and also possible as long as the opponent is blocked from expanding.

	An expansion is desired by the Economist when all mineral fields have been saturated, as otherwise economic growth becomes impossible. Before this, it is sometimes necessary depending on the match, but this is difficult to determine. This expansion might be too late in some cases, but will at least ensure a consistent worker production throughout the game.

	There have been tested three versions of bot with regards to expansion. The initial version never expanded, where the second version would always expand whenever desirable as described before. The third version would toggle between the behaviors, turning expansion behavior on if the opponent built defensive structures. These would usually only be used early in a turtle strategy, in which case the opponent is passive and expansions are less risky. The results and their discussion can be found in Chapter \ref{ch:results}.

	An earlier version of The bot used a fast expand opening, but it was less successful compared to its usual one base play. The fast expand is not always viable, and as such requires some modeling of the opponents' strategies to determine. It was down-prioritized in favor of the more versatile saturation expansion.
	
	Since expanding is expensive, the Economist needs to save up resources. As mentioned in Section \ref{sec:buildOrder}, an expansion is built by queuing it in the Planner, pausing other productions until it has been built.
	
	\subsection*{Maynard Slide}
	Once an expansion has been built, it will initially have no workers at its resources. The optimal solution is have all workers equally scattered across all resources by depots, since this ensures maximum return value for each worker. If a base reaches maximum worker capacity, it can no longer contribute to worker production efficiently since completed workers will have to be moved. Additionally, older bases will not deplete resources as quickly and new bases will return their investments immediately.
	
	Therefore, upon building a new depot, players will equalize the workforce between bases by moving workers to the new expansion, called a \emph{Maynard Slide}. This has been subject to some statistical analysis, as it is a quite common strategy in StarCraft. When moving workers, there must remain at least one per resource since they already have a maximum return rate.
	
	This is implemented in the Economist, which is notified by BWAPI once a new depot has been constructed. Iterating through vassals in a random order, it moves workers beyond the first per resource to the new base, until it either runs out of available workers or reaches two per mineral. It is not concerned with gas harvesting, as this did not see use.

	\subsection*{Building Supply}
	Recall the \emph{supply} resource explained in Chapter \ref{ch:resources}. A player can have a maximum of 200 supply, and it is required by almost every non-building unit in the game. The challenge issued to players is being the least possible \emph{supply-blocked}, that is having reached the supply limit, as this will block production.
	
	The simplest supply implementation is to build a supply unit whenever the limit has been reached. Production will however still be stopped while the supply structure is being constructed. Building supply before the limit has been reached will shorten or remove the pause, which can allow for unbroken production depending on the available resources. If the supply is built too early, a lack of resources might also block production. The optimal solution is where there is the least amount of time in which units are not produced. Observe however that this is alike the \emph{Job shop scheduling} problem, but more complex with the additional factor of resources. As such the problem is NP-complete, even if future resource intake was predicted.

	An improved version of the simplest solution, implemented in the Economist, is to build supply when the current limit is below reserved supply plus a constant. This somewhat works, as the agent quickly reaches a specified amount of unit producers in each game, thus requiring the same buffer. The constant was found through some testing, where the result is the supply is built a bit too early. This however impacts performance the least beyond an optimal solution, and has proven sufficient.
	
	The same solution could be made dynamic by making the constant variable according to predicted supply needs. Since the required construction time for the supply unit is known, the Economist could predict how many units will be produced during this time, and set the variable accordingly. The prediction could be done without accounting for resource costs for simplicity.
	
\section{Combat Strategy}
Building structures and training units for combat is handled by the \emph{Strategist}. This module decides which units become available for the Attacker and Defender, and therefore controls the overall strategy of the bot. As the bot is using a rush strategy, the goal of the Strategist is to train as many troops as quickly as possible. There is a balance between producing trainers and troops, as having multiples of the former helps the latter. Too many troop producers might reach more troops faster, but spends a lot of resources that could have become more troops. Too few producers, and the rush will be too late.

Rarely does the first rush win the game, the rushing player instead keeps pressuring the opponent with attacks. The goal of each attack is to either replace lost troops faster or destroy more units than the opponent, until they are overwhelmed.

	\subsection*{Building Troops}
	The first unit the Protoss has access to is the \emph{zealot}. Compared to the other units first available to the other races, the zealot is much more expensive but also a lot stronger. It is known for being very cost effective and a few of them are able to beat other races' rush attempts before more advanced units are available. This makes them ideal for rushing, even though this is seldom used in human competition, it has proven very effective against bots, though the reason is unclear.
	
	Commanding units effectively is very difficult, and as Chapter \ref{ch:combat} described, this agents grasp on units in combat is simple. Fortunately, the zealot's brute force strength, coupled with regenerating shield and melee attacks makes it easy to control effectively. In favor of avoiding the use of much more complex units, the Strategist only trains zealots.
	
	This does have issues. While they are cost effective, they are still out-performed more advanced units, especially air- and cloaked units which they cannot attack. However, these units are only available later in the game, at which point the agent will have amassed a sizable army of zealots and attacked.
	
	As this bot is very simple in terms of handling units, it is appropriate to use the first available units and attack as quickly as possible. The idea behind this strategy is, that since this bot lacks greater strategic reasoning and handles late game units poorly, then it should attempt to win the game before any of that becomes relevant. This can beat more complex bots, simply because their otherwise superior strategies are never allowed to hatch. Additionally, Protoss has the strongest and easiest to use early game units, and when massed will easily beat both Terran and Zerg.

	The Strategists' production logic is just a greedy algorithm, attempting to produce more zealots each frame.

	Against static defenses the zealots fare poorly. This is where the bot attempts to expand and instead overwhelm the opponent in zealots, which has mixed results. The economic advantage can sometimes be enough to outweigh the technological disadvantage. An earlier version of the bot attempted to use the second available Protoss unit, the \emph{Dragoon}, but it was too inefficient with the simple combat techniques. The expansion feature was prioritized instead.
	
	\subsection*{Building Troop Producers}
	Since the Strategist only uses one unit, it only needs one kind of structure called the \emph{gateway}, which can train zealots. Again the Strategist uses a greedy algorithm, attempting to build more gateways every frame until it reaches a set threshold of troop producers.
	
	It was found through testing that the starting location, with maximum worker capacity, can support four gateways constantly producing zealots. This gets more difficult to asses with expansions, and through testing it was found that greedily building limitless gateways was not efficient. The current limited solution is building four gateways without expansions, and six if there are any expansions. A more scalable solution has not been implemented as the game usually does not reach further.